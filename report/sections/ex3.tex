\documentclass[../main.tex]{subfiles}


\begin{document}
\section{Anderson Model}

The Anderson Model was one of the first models that incorporated randomness when it was first proposed.
It tries to model systems that have irregular or unforeseeable impurities in the Hamiltonian.
\par

Generally the Anderson Model implements a tight-binding Hamiltonian with next-neighbor hopping.
In addition to the hopping mechanism we add a small random potential at every lattice site.
The microscopic Hamiltonian is 
\[
    \mathcal{H} = t\sum\limits_{\left<i j \right>}^{ } c_i^{\dagger} c_j + \text{h.c.}  + \sum_{i}^{ }  E_i c_i^{\dagger} c_i, 
    \qquad \text{where} \qquad
    \rho(E) = \frac{1}{W}\Theta\left( \frac{W}{2} - E \right)
.\] 

When randomly introduced disorder becomes too large in this system, the system starts to localize and particles stop diffusing throughout the lattice.
In this part of the Lab Course we want to study this process with the methods that we have acquired from the previous exercises.


\subsection{Spectrum}

From the spectral density of the system we can interpolate it's properties:
In the tight-binding model we find $\rho(E=E_0) \ll 1$ ($E_0 \approx 0$ in the Anderson Model) while the edges of the density of states $E_0 \pm 2t$ are much more frequently occupied.
Using our methods we can calculate the spectral density of the system, which in turn is very closely related to the density of states.
Such we can use our methods to analyze the localization behavior of the model.


\subsection{}



\ifSubfilesClassLoaded{
	% if it's compiled alone
}{
	% if it's compiled in the main file
    \newpage
}
\end{document}

